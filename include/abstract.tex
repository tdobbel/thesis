
\chapter*{Abstract}
%\phantom{Abstract}

Coral populations have declined dramatically worldwide under the combined effects of climate change and local anthropogenic stressors. In addition to ocean warming and acidification, threats to coral reefs in the Caribbean include frequent and virulent disease outbreaks, and intensifying hurricanes. In particular, Florida's Coral Reef (FCR) is currently facing a multi-year outbreak of the stony coral tissue loss disease (SCTLD). First observed in 2014 during the monitoring of the PortMiami Deep Dredge Project (PMDDP), the disease has now spread through the entire FCR and has been reported in numerous territories of the Caribbean. However, the propagation of the disease through FCR seemed to slow down when it reached its southwestern end, between the Marquesas and the Dry Tortugas (DRTO). Although the causative agent of the disease remains unknown, studies showed that its transmission was likely waterborne and that sediments could act as SCTLD vectors. The hydrodynamics should therefore be highly explanatory of its spread. Furthermore, Florida is a prime landfall target for hurricanes during the reproduction period of corals. In addition to causing wholesale destruction of the reefs, hurricanes might thus also impact larval dispersal through wind-wave-induced currents. Here, we use the high resolution coastal ocean model SLIM to capture transport processes in the FCR, and study the impact of SCTLD and hurricanes to coral reefs. First, a coupled epidemio-hydrodynamic model is developed to reproduce and understand the observed spread of SCTLD in Florida. A sediment transport model is then used to evaluate the impact of the PMDDP on the observed onset of the disease in 2014. Finally, a coupled wave-current model is implemented to study the impact of Hurricane Irma (2017) on transport processes over reefs in the Florida Keys. Assuming the dispersal of disease agents within neutrally buoyant material, our model successfully reproduced the observed spread of SCTLD through the FCR and linked its apparent stalling before reaching the DRTO to eddy activity in the Loop Current/Florida Current system. Furthermore, the results of our sediment transport model suggest that the PMDDP might have triggered the onset of the disease. Finally, our coupled wave-current model showed the important impact of wave-induced currents on transport processes over reefs during hurricanes. This thesis highlights the potential of models informed by and confronted against field knowledge as powerful tools to inform the management of complex marine ecosystems such as the FCR.