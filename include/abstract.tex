
\chapter*{Abstract}
%\phantom{Abstract}

Coral populations have declined dramatically worldwide under the combined effects of climate change and local anthropogenic stressors. In addition to ocean warming and acidification, threats to coral reefs in the Caribbean include frequent and virulent disease outbreaks, and intensifying hurricanes. In particular, Florida's Coral Reef (FCR) is currently facing a multi-year outbreak of the stony coral tissue loss disease (SCTLD). First observed in 2014 during the monitoring of the PortMiami Deep Dredge Project (PMDDP), the disease has now spread through the entire FCR and has been reported in numerous territories of the Caribbean. However, the propagation of the disease through FCR seemed to slow down when it reached its southwestern end, between the Marquesas and the Dry Tortugas (DRTO). Although the causative agent of the disease remains unknown, studies showed that its transmission was likely waterborne and that sediments could act as SCTLD vectors. The hydrodynamics should therefore be highly explanatory of its spread. Furthermore, Florida is a prime landfall target for hurricanes during the reproduction period of corals. In addition to causing wholesale destruction of the reefs, hurricanes might thus also impact larval dispersal through wind-wave-induced currents. Here, we use the high resolution coastal ocean model SLIM to capture transport processes in the FCR, and study the impact of SCTLD and hurricanes to coral reefs. First, a coupled epidemio-hydrodynamic model is developed to reproduce and understand the observed spread of SCTLD in Florida. A sediment transport model is then used to evaluate the impact of the PMDDP on the observed onset of the disease in 2014. Finally, a coupled wave-current model is implemented to study the impact of Hurricane Irma (2017) on transport processes over reefs in the Florida Keys. Assuming the dispersal of disease agents within neutrally buoyant material, our model successfully reproduced the observed spread of SCTLD through the FCR and linked its apparent stalling before reaching the DRTO to eddy activity in the Loop Current/Florida Current system. Furthermore, the results of our sediment transport model suggest that the PMDDP might have triggered the onset of the disease. Finally, our coupled wave-current model showed the important impact of wave-induced currents on transport processes over reefs during hurricanes. This thesis highlights the potential of models informed by and confronted against field knowledge as powerful tools to inform the management of complex marine ecosystems such as the FCR.

%The population of corals throughout the world has declined dramatically and many coral species are now listed as threatened Avoiding global warming being no longer an option, reef managers now focus on damage control and restoration strategies. The Florida Reef Tract, currently facing an unprecedented coral disease outbreak and where healthy corals grown in nurseries are replanted on damaged reefs, is a good application case of such reef management strategies. In order to maximize the efficiency of these efforts, a good knowledge of exchanges of infectious materials and larvae between reefs is required. However, connectivity patterns between reefs remain mostly unknown. To address this issue, we will use the high-resolution ocean model SLIM to simulate connectivity patterns between reefs and then identify reefs with the highest restoration potential as well the ones that are most at risk of being infected by the outbreak. 


%The dispersal of coral larvae away from their natal habitats is an important process for coral reef ecosystems, but remains poorly understood and hard to gauge. This thesis presents a spatially explicit numerical modelling approach to simulate larval dispersal in Australia's Great Barrier Reef (GBR), and in doing so to estimate connectivity between the thousands of reefs that comprise it. A high resolution, finite element ocean model, SLIM, is used to build a biophysical model of larval dispersal capable of estimating larval exchange between reefs in large sections of the the GBR. Tools from network science are employed to investigate the community structure of the GBR's connectivity network, and these tools are used to reveal the presence of clusters of highly inter-connected reefs. Subsequently, the model is used to estimate the extent of connectivity between reefs of different morphologies and depths in the GBR. Finally, the possible impacts of climate change on connectivity in the GBR are assessed, and estimates are made of the likely changes to coral larval dispersal patterns over the coming century.

%Titan, Saturn’s largest moon, is a unique object in the Solar system as much as it has a substantial atmosphere, a surface with a complex interplay of geological processes and an outer ice shell overlyinga subsurface ocean. The climate on this icy satellite boasts a multi-phase hydrological cycle where methane plays a role similar to that of water on Earth. Surface lakes and seas filled with liquid methane and ethane are found in the polar regions. The subsurface ocean, on the other hand, is filled with liquid water. This liquid layer lying beneath Titan’s surface allows for the large surface deformation observed.

%The first objective of this thesis was to adapt an Earth-based geophysical and environmental model, SLIM (www.climate.be/slim), to Titan’s specific conditions in order to study the tidal motion in the surface lakes and seas. The modified model was applied to the largest lake in the southern hemisphere, Ontario Lacus, and the two largest seas, Kraken and Ligeia Maria, which are located in the northern hemisphere. The predicted surface elevation and velocity fields are part of the data needed to develop an exploration mission focusing on the surface lakes and seas of Titan. The normal modes of Ontario Lacus were also numerically studied. While resonantly forced modes could generate significant liquid motion, the natural periods are much shorter than the period of the astronomical forcings. Therefore, only atmospheric forcings could resonantly force the normal modes. Strong wind conditions corresponding to a stormy event are required to generate a surface elevation resulting in significant shoreline variations.

%Then, the model was used to study the tides of Titan’s global subsurface ocean. To this end, the shallow water equations and the 3D hydrostatic equations under the Boussinesq approximation were modified in order to take into account the ice shell lying at the top of the ocean. The method was adapted to solve the equations on a sphere. The effects of the shell are represented by adding a surface friction and a surface pressure term. The shell decreases the ocean surface elevation and slows down the flow without modifying significantly the global patterns of these fields. The magnitude of these variations depends on the mechanical behaviour of the ice shell. Surface and bottom heat fluxes play a significant role in the liquid motion of the ocean. The interactions between the tidal motion and the thermally driven flow resulting from the surface heat flux were studied by means of the 3D version of SLIM. The surface heat flux significantly impacts the velocity field, both in terms of magnitude and orientation while the influence on the ocean’s surface elevation is small. These results could be useful for astrobiologists paying attention to the ocean habitability and to validate hypotheses about Titan’s internal structure.