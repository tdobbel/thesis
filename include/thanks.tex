 \phantom{thanks}
%\textit{Remerciements à }
\chapter*{Remerciements}

Mes premiers remerciements vont à mon promoteur, le professeur Emmanuel Hanert, le grand chef qui m'a guidé et accompagné tout au long de cette thèse. D'une énergie et surtout d'un optimisme à toute épreuve (si toutes ses prédictions s'étaient avérées exactes, cette thèse aurait été finie en 1 ans), il m’a poussé et soutenu durant les quatre dernières années. En outre, il a toujours réussi à accomplir le miracle de dégager du temps dans son agenda bien chargé pour me conseiller ou relire et commenter mon travail avec minutie. Je tiens également à remercier tous les membres de mon jury pour le temps qu'ils ont accordé à la lecture de ce manuscrit ainsi que pour leurs judicieux conseils et suggestions lors de la défense privée. 

Je remercie également le groupe de vie ENGE, qui m'a accueilli durant les bientôt quatre années de cette thèse. Ce groupe hétéroclite m'a fourni un cadre de travail convivial, agrémenté de pauses café, de verres à la quinzaine et de soupers de Noël endiablés. Ce fut un plaisir de les avoir comme collègues et je ne pourrai désormais plus manger de burrito sans penser à eux. Parmi eux, je tiens particulièrement à remercier mon fidèle voisin de bureau et excellent camarade de conférence, Antoine Saint-Amand. Il m'a guidé et accompagné au cours de mon épopée dans la jungle de la thèse et a été une source intarissable de bonnes histoires et anecdotes. De plus, son engagement à la ville comme à l’université ainsi que son dévouement pour ses étudiants ont été une source d’inspiration. Il me reste à remercier l’autre mouscronnois du bureau, Valentin Vallaeys, pour son humour et sa propension à inonder ses collègues de sucreries. En outre, son don inégalé pour poser les bonnes questions m'a plus d'une fois permis d’entrevoir une issue à un problème apparemment insoluble l’instant d’avant.

Je tiens également à remercier tous les autres membres de la team SLIM. Ce fut un plaisir de travailler avec eux au développement de ce modèle. Parmi eux, je tiens à remercier spécialement Jonathan Lambrechts, qui rend l’impossible possible en 3 coups de clavier. Malgré les nombreuses sollicitations qu’il reçoit, il s'est toujours rendu disponible pour me débugger et cette thèse aurait été techniquement impossible sans lui. À ce propos, un échange téléphonique particulièrement éclairant me revient en mémoire: “Salut Jon, est-ce que tu as un peu de temps pour m’aider ? Non, mais dis toujours.”... 5 minutes plus tard, le problème était réglé.

I would also like to thank the team of collaborators and co-authors that started in 2019 with Lew Gramer, later joined by Erinn Muller and Dan Holstein, and has been growing ever since. It has been a real pleasure to work with them on all these exciting projects. I would like to extend my thanks to Pr Sam Purkis and Ceci Lopez-Gamundi. Although unraveling the mysteries of the Great Bahama Bank was not directly linked to the subject of this thesis, it has been a pleasure to join them in this geological quest. Et tant qu’on est à Miami, je remercie également Matthieu le Hénaff, qui garde un œil sur nos projets floridiens en coulisse. Ses conseils avisés ainsi que les personnes avec lesquelles il nous a mis en contact ont été d’un grand secours.

Il convient également de remercier ceux qui n'ont probablement pas contribué à me rendre plus productif mais ont assurément rendu ces années de thèse beaucoup plus amusantes. Tout d'abord, il me faut remercier les 36 "voix" du kot Carrefour. Cette grande famille m'a permis de rencontrer des gens d'origines et de cultures variées tout en me fournissant un lieu de vie calme (à l’exception de quelques folies) pour mener à bien cette thèse. Les prochains remerciements sont pour l’équipe de l'Oenokot, qui réussit à convaincre tout le monde (même mes anciens étudiants) que je suis un étudiant en BAC2 de droit. Des remerciements tout particuliers s'imposent pour le dénominateur commun à ces deux bandes de joyeux lurons, Redge, qui me supporte depuis presque 5 ans malgré des premiers contacts difficiles au kot Erasmus. Enfin, je remercie la Revue des Ingénieurs, qui offre un auditoire à mes blagues depuis 9 ans et qui, jusque dans les derniers jours de la rédaction de cette thèse, m’a fourni quelques bons prétextes à la procrastination.

\vspace{3em}

Ce travail a été financé par le Fonds de la Recherche Scientifique - FNRS au moyen d’une bourse FRIA.
