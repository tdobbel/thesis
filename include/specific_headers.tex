%\usepackage[includehead, includefoot, top=2.35cm, bottom=2.35cm, outer=3.5cm, inner=3.5cm]{geometry}
\usepackage[includehead, includefoot, top=4.8cm, bottom=4.8cm, outer=4.5cm, inner=4.5cm]{geometry} % l'option showframe est juste pour avoir le cadre défini par les bordures (pour voir ce qui dépasse dans les figures ou equations par ex.) 
\usepackage{bm}			%For bold symbols in math (command \bm)
\usepackage{lmodern}		%Font: latin modern (enhanced computer modern)


\usepackage{url}
\usepackage{threeparttable}

\usepackage{arydshln}

\usepackage[font=small]{caption}

\usepackage{color,calc,graphicx}
\newsubfloat{figure}% Allow subfloats in figure environment

\newcommand\red[1]{\textcolor{red}{#1}}
\newcommand\blue[1]{\textcolor{blue}{#1}}

\usepackage{geometry}
\newdimen\mywidth
\mywidth=0.91\textwidth
\newdimen\mywidthmesh
\mywidthmesh=0.80\textwidth
%\setlength{\belowcaptionskip}{-20pt}
\usepackage{siunitx} % for micro meters

% \usepackage{tikz}
% % \input{include/arrowsNew}
% \usetikzlibrary{shapes.misc}
% \tikzset{options/.code={\tikzset{#1}}} % just to compact the code
% \tikzset{cross/.style={cross out, draw=black, minimum size=2*(#1-\pgflinewidth), inner sep=0pt, outer sep=0pt},
% %default radius will be 1pt. 
% cross/.default={1pt}}

% \usetikzlibrary{decorations.pathmorphing, shadows}
% %\usetikzlibrary{arrows}
% \usetikzlibrary{shapes}

% \usetikzlibrary{positioning,calc,patterns} 
%\usetikzlibrary{external}
%\tikzexternalize[prefix=tikzext/] 

\usepackage{algorithm2e}


%For the glossary:
\usepackage{float}
% % Create new "listing" float
% \newfloat{listing}{tbhp}{lst}%[section]
% \floatname{listing}{Listing}

%Allow big floats:
\renewcommand{\topfraction}{.95}
\renewcommand{\bottomfraction}{.8}
\renewcommand{\textfraction}{.05}		%Allow minimal text with figures
\renewcommand{\floatpagefraction}{.85}		% require fuller float pages
	% N.B.: floatpagefraction MUST be less than topfraction !!
\renewcommand{\dbltopfraction}{.66}
\renewcommand{\dblfloatpagefraction}{.8}

% Command to set caption styles
\usepackage{etoolbox} 		%Needed for AtBeginEnvironment
\captionnamefont{\bfseries\small}
\captiontitlefont{\small}
\subcaptionlabelfont{\bfseries\small}
\subcaptionfont{\footnotesize}

\renewcommand{\figurename}{Fig.}

%For subappendices:
\usepackage{appendix}
\usepackage{chngcntr}
\AtBeginEnvironment{subappendices}{%
  \chapter*{Appendix}
  \addcontentsline{toc}{chapter}{Appendices}
  \counterwithin{figure}{section}
  \counterwithin{table}{section}
}
\AtEndEnvironment{subappendices}{%
  \counterwithout{figure}{section}
  \counterwithout{table}{section}
}

\renewcommand{\sfdefault}{lmss}
\usepackage{epstopdf}
\usepackage{fancybox}
\usepackage{booktabs}		% For \...rule lines in tables

%\usepackage[style=authoryear, 
%  sorting=nyt, 
%  sortcites=true, 
%  firstinits=true, 
%  uniquename=false, %2 lines to avoid initials in citations
%  uniquelist=false, %
%  doi=false, 
%  isbn=false, 
%  autopunct=true, 
%  maxcitenames=2,
%%   mincitenames=1,
%  maxbibnames = 1000,
%  uniquelist = false,		%Make 2008a, 2008b
%  backend=biber]{biblatex}

\usepackage{natbib}
\bibliographystyle{model2-names}


\usepackage{amsmath}
% \usepackage{amsfonts}
\usepackage{textcomp}						%For \textdegrees
\usepackage{verbatim}
\usepackage{amssymb}
% \usepackage{amsthm}
\usepackage{array}

% Enumerate with letters
\usepackage{enumerate}


\usepackage{empheq}

% \usepackage[round]{natbib}




\definecolor{red}{rgb}{1,0,0}
\definecolor{blue}{rgb}{0,0,0.8}
\definecolor{green}{rgb}{0,0.5,0}
\definecolor{black}{rgb}{0,0,0}
%\newcommand{\emphc}[1]{}
\newcommand{\emphc}[1]{\emph{\textcolor{red}{#1}}}
\newcommand{\emphb}[1]{\emph{\textcolor{blue}{#1}}}
\newcommand{\modif}[1]{\textcolor{black}{#1}}







%
%%
%\usepackage{draftwatermark}
%\SetWatermarkText{DRAFT}
%\SetWatermarkScale{1}





%%%%%%%%%%% CONGO2D %%%%%%%%%

\usepackage{nicefrac}
\usepackage{xspace}
\usepackage{multirow}
% \newcommand{\gmsh}{{\textsc gmsh}\xspace}
% \newcommand{\rofi}{{\textsc rofi}\xspace}
\newcommand{\slim}{{\textsc slim}\xspace}
\newcommand{\UV}{\mathbf{U}}
\newcommand{\hycom}{\textsc{hycom} }
\newcommand{\ie}{\textit{i.e}} 
\newcommand{\eg}{\textit{e.g.}}

\DeclareSIUnit \year{y}
\DeclareSIUnit \cmetre{\metre\textsuperscript{\nicefrac{\num{-1}}{3}}}
\DeclareSIUnit \pmetre{\metre\textsuperscript{\num{0.85}}}
\DeclareSIUnit \centimetre{\centi\metre}


\newcommand{\clearemptydoublepage}{\newpage{\pagestyle{empty}\cleardoublepage}} 
\newcommand{\mychapter}[1]{\clearemptydoublepage \chapter{#1}}
\newcommand{\mychapterb}[2]{\clearemptydoublepage \chapter[#1]{#2}}
\newcommand{\mypart}[1]{\clearemptydoublepage \part{#1}}
\newcommand{\chapternonumber}[1]{\clearemptydoublepage \chapter*{#1 \markboth{#1}{}}\addcontentsline{toc}{chapter}{#1}}
\newcommand{\sectionnonumber}[1]{\section*{#1 \markright{#1}}\addcontentsline{toc}{section}{#1}}
\newcommand{\partnonumber}[1]{\clearemptydoublepage \part*{#1 \markboth{#1}{}}\addcontentsline{toc}{part}{#1}}

\newcommand{\bigcell}[2]{\begin{tabular}{@{}#1@{}}#2\end{tabular}} % For multiline cells inside tables

%Set headers at the top of pages
\nouppercaseheads
\makepagestyle{mystyle}
\makeevenhead{mystyle}{\thepage}{}{\itshape\leftmark}
\makeoddhead{mystyle}{}{}{\thepage}
\makeevenfoot{mystyle}{}{}{}
\makeoddfoot{mystyle}{}{}{}
\makepsmarks{mystyle}{%		Define what's in headers, see p12 of https://www.tug.org/pracjourn/2008-2/madsen/madsen.pdf
%   \createmark{section-heading-to-use}{define-for-left-or-right-side}{show-sec-nb?}{prefix-to-secNb}{suffix-to-secNb}
  \createmark{chapter}{left}{shownumber}{Chapter }{ - \ } 
}
% Nb: if I have appendices, will need to create new style so it isn't prefixed by Chapter

%   \makeheadrule{mystyle}{\textwidth}{\normalrulethickness}
% Extend the page numbers (& horiz line) into the margin
\makerunningwidth{mystyle}{1.\textwidth}
\makeheadposition{mystyle}{flushright}{flushleft}{flushright}{flushleft}
\makeheadrule{mystyle}{1.\textwidth}{\normalrulethickness}

\pagestyle{mystyle}

%Ref a footnote
\makeatletter
\newcommand\footnoteref[1]{\protected@xdef\@thefnmark{\ref{#1}}\@footnotemark}
\makeatother

%Set the style for the chapter headings
%\usepackage{fourier} %% I COMMENTED THIS!!
\DeclareSymbolFont{calletters}{OMS}{cmsy}{m}{n}
\DeclareSymbolFontAlphabet{\mathcal}{calletters}
%\usepackage[upright]{fourier}
\usepackage{soul}
%\definecolor{nicered}{rgb}{.647,.129,.149}
\definecolor{nicered}{rgb}{0.5,0.5,0.5}
\makeatletter
\newlength\dlf@normtxtw
\setlength\dlf@normtxtw{\textwidth}
\def\myhelvetfont{\def\sfdefault{mdput}}
\newsavebox{\feline@chapter}
\newcommand\feline@chapter@marker[1][4cm]{%
  \sbox\feline@chapter{%
     \resizebox{!}{#1}{\fboxsep=1pt%
       \colorbox{nicered}{\color{white}\bfseries\sffamily\thechapter}%
     }}%
  \rotatebox{90}{%
     \resizebox{%
       \heightof{\usebox{\feline@chapter}}+\depthof{\usebox{\feline@chapter}}}%
     {!}{\scshape\so\@chapapp}}\quad%
  \raisebox{\depthof{\usebox{\feline@chapter}}}{\usebox{\feline@chapter}}%
}
\newcommand\feline@chm[1][4cm]{%
  \sbox\feline@chapter{\feline@chapter@marker[#1]}%
  \makebox[0pt][l]{% aka \rlap
     \makebox[1cm][r]{\usebox\feline@chapter}%
  }}
\makechapterstyle{daleif1}{
  \renewcommand\chapnamefont{\normalfont\Large\scshape\raggedleft\so}
  \renewcommand\chaptitlefont{\normalfont\huge\bfseries\scshape}%\color{nicered}}
  \renewcommand\chapternamenum{}
  \renewcommand\printchaptername{}
  \renewcommand\printchapternum{\null\hfill\feline@chm[3cm]\hspace*{1cm}\par}
  \renewcommand\afterchapternum{\par\vskip\midchapskip}
  \renewcommand\printchaptertitle[1]{\chaptitlefont\raggedleft ##1\par}
}
\makeatother
\chapterstyle{daleif1}

\setlength{\bibsep}{0.0pt}

\definecolor{darkgreen}{rgb}{0.0,.6,0.0}
\newcommand{\todo}[1]{\textcolor{red}{TO DO: #1}}