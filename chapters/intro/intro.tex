\lipsum[1-12]
\clearpage

\section*{Supporting publications}

%The research carried out in the framework of this doctoral thesis has led to the following articles:

\begin{list}{}{%
\setlength{\topsep}{0pt}%
\setlength{\leftmargin}{0.23in}%
\setlength{\listparindent}{-0.23in}%
\setlength{\itemindent}{-0.23in}%
\setlength{\parsep}{\parskip}%
}%


\item \textbf{Dobbelaere T.}, E.M. Muller, L. J. Gramer, D. M. Holstein and E. Hanert (2020),  Coupled Epidemio-Hydrodynamic Modeling to Understand the Spread of a Deadly Coral Disease in Florida. Frontiers in Marine Science, 7, 1-16.

\item \textbf{Dobbelaere T.}, M. Curcic, M. Le Hénaff and E. Hanert (2021), Impacts of Hurricane Irma (2017) on ocean transport processes, Ocean Modelling, accepted.

\item \textbf{Dobbelaere T.}, D. M. Holstein, E.M. Muller, L. J. Gramer, L. McEachron, S.D. Williams and E. Hanert (2021), Connecting the dots: Transmission of stony coral tissue loss disease from the Marquesas to the Dry Tortugas. Frontiers in Marine Science, accepted.

\item Purkis S.J., A.M. Oehlert, \textbf{T. Dobbelaere}, E. Hanert and P. Harris (2020), Always a White Christmas in the Bahamas - Ocean Chemistry and Hydrodynamics Focus Winter Mud Production on Great Bahama Bank, Sedimentology, in revision.

\item Alaerts L., \textbf{T. Dobbelaere}, P.M. Gravinese and E. Hanert (2022), Climate change will fragment Florida stone crab communities. Frontiers in Marine Science, submitted.

\item Hanert E., Aboobacker V.M., Veerasingam S, \textbf{Dobbelaere T.}, Vallaeys V. And E. Hanert (2021) A multiscale ocean modelling system for the central Arabian Gulf: From basin-scale circulation patterns to flow-structure interactions. Estuarine, Coastal and Shelf Science, submitted.

\item Anselain T., \textbf{T. Dobbelaere}, E. Heggy and E. Hanert (2022) Qatar oil spill vulnerability threatens both its national water security and the global gas market. Nature Communications, in preparation.

\end{list}

\section*{Public presentations}

%The research carried out in the framework of this doctoral thesis has led to the following articles:

\begin{list}{}{%
\setlength{\topsep}{0pt}%
\setlength{\leftmargin}{0.23in}%
\setlength{\listparindent}{-0.23in}%
\setlength{\itemindent}{-0.23in}%
\setlength{\parsep}{\parskip}%
}%

\item \textbf{D. Vincent}, O. Karatekin, V. Dehant, E. Deleersnijder, 2015 Numerical simulations of tides in Ontario Lacus. European geoscience Union - General Assembly, Vienna. \\http://hdl.handle.net/2078.1/158392

\item \textbf{D. Vincent}, O. Karatekin, V. Dehant, E. Deleersnijder, 2016, Modelling of tidal flows between Titan's seas Kraken Mare and Ligeia Mare. Titan surface meeting, Paris.

\item \textbf{D. Vincent}, O. Karatekin, V. Dehant, E. Deleersnijder, 2016, Modelisation of tidal flows between Titan's seas Kraken Mare and Ligeia Mare. European geoscience Union - General Assembly, Vienna.\\
http://hdl.handle.net/2078.1/173566

\item \textbf{D. Vincent}, O. Karatekin, V. Dehant, E. Deleersnijder, Eric, 2017, Study of liquid exchanges between Titan's seas Kraken Mare and Ligeia Mare. European geoscience Union - General Assembly, Vienna. \\
http://hdl.handle.net/2078.1/185118

\item \textbf{D. Vincent}, O. Karatekin, V. Dehant, E. Deleersnijder, Eric, 2017, Modelling the tides of Titan's liquid bodies. TECLIM Seminar, Louvain-la-Neuve.

\item \textbf{D. Vincent}, O. Karatekin, J. Lambrechts, V. Dehant, E. Deleersnijder, 2018, Tides and eigenmodes of an idealized subsurface global ocean on Titan. European geoscience Union - General Assembly, Vienna. \\
http://hdl.handle.net/2078.1/196902

\item H. A. Le,X. H. D. Vu, J. Lambrechts, S. Ortleb, V. Vallaeys, \textbf{D. Vincent}, N. Gratiot, S. Soares Frazao, E. Deleersnijder, 2018, A new wetting – drying algorithm integrated in SLIM, with an application to the Tonle Sap lake, Mekong. European geoscience Union - General Assembly, Vienna.\\ http://hdl.handle.net/2078.1/196904







\end{list}
