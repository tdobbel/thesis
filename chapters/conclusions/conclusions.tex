\section{Conclusions}
The previous chapters of this thesis have presented an effort to build a modeling framework to better understand the propagation and onset of the SCTLD, as well as the impacts of major hurricanes on the transport processes over coral reefs. This model had to be able to accurately capture ocean circulation at different scales to study transport processes in the FRT. First, the large-scale Loop Current-Florida Current system had to be accurately reproduced to correctly capture its impact on eddy formation near the DRTO and the Florida Keys. Second, the model had do resolve ocean circulation at the reef-scale to capture recirculation and acceleration of currents between reefs and islands. This was performed by coupling the multi-scale ocean model SLIM to a connectivity-based epidemiological model (chapter \ref{chap:sctld}), a sediment transport model (chapter \ref{chap:onset}) and a spectral wave model (chapter \ref{chap:irma}).

The developed coupled hydro-epidemiological model reproduced the observed propagation of the SCTLD through the FRT by modeling the transport of its causative agent within neutrally buoyant material driven by mean barotropic currents. It confirmed the result of previous ex situ studies that showed evidence of waterborne transmission of the SCTLD with an averaged transmission time of the order of 10 days. Furthermore, by using coral resistance to the disease as a parameter, our model results suggested that, on average, corals had pretty low resistance against the disease. This confirmation of experimental results by model results illustrates the important potential of models in evaluating hypotheses derived based field observations on large scale systems. As such, models informed and confronted against field knowledge are a powerful tool to inform the management of complex ecosystems such as a barrier reef.

This epidemiological model was then used to build networks of potential exchanges of disease agents between reefs during a period of three years. The analysis of these networks showed that exchanges from the western end of the DRTO were interrupted during most of 2020. This interruption was likely linked to eddy activity near the DRTO and was consistent with the apparent stalling of the outbreak in the region in 2020. This confirms the role played the Loop Current/Florida Current system in the control of the connectivity in the Florida Keys and the DRTO. Furthermore, location predicted to be more vulnerable to SCTLD in the disease networks were consistent with the observed geographic distribution of disease coral in the DRTO. This further highlights that modeled hydrodynamics are highly explanatory of the propagation of the disease.

The results of chapters \ref{chap:sctld} and \ref{chap:drto} show that a high-resolution biophysical is a valuable tool to understand the spread of SCTLD in Florida. The model successfully reproduced the observed propagation of the disease in 2018-2019 and allowed to link the observed stalling of SCTLD to hydodynamic features. 

\section{Perspectives for future works}

\subsection*{Improvement of the epidemiological model}


\subsection*{Better understanding the protective role of corals against storm waves}