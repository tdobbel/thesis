%\linespread{1.5}\selectfont


The astronomers' interest for Titan was raised by the observations of Voyager spacecrafts. Further explorations were conducted by means of Earth-based observation tools and the successful Cassini mission provided plentiful data whose analysis is an ongoing process. While this mission answered a variety of questions, crucial ones remain open. To this end, various exploration missions have been designed. One of them, Dragonfly \citep{lorenz2018dragonfly}, was recently selected by the NASA as part of its New Frontiers program while others are still under development and are proposed to several spatial agencies \citep{mitri2014exploration,hartwig2016exploring}. Despite the numerous projects, new \textit{in situ} data are not expected before at least 14 years. Indeed, Dragonfly, which is the most advanced project, is expected to be launched in 2026 and will land in 2034. Until then, Earth-based observations, data analysis by means of enhanced algorithms, and numerical modelling are the only tools able to provide new pieces of information about Titan.



\section*{Exploration of Titan's liquid bodies}

Titan is the only celestial body of the solar system, besides the Earth, housing significant surface liquid bodies. The latter played a key role in the apparition of life on our planet. It was thus instinctive to pay interest to the surface lakes and seas of Titan although the latter are filled with liquid hydrocarbons instead of water. The oceanography of Titan is still in its early stages due to the lack of data and observations, which leads to large uncertainties in the models predictions and prevents the scientists from validating their model(s) by comparison with \textit{in situ} measurements. Unfortunately, pieces of information only available through numerical simulations are requested to design an exploration mission focusing on the surface lakes and seas. For instance, an \textit{a priori} knowledge of the flow is a key feature in the design of an exploration mission. It is a chicken-and-egg problem. To circumvent this paradox, one has to proceed in stages. The first step consists in running numerical simulations of the flow in the surface lakes and seas using the scarce data available. Depending on the landing craft, different aspects of the liquid motion need to be known. For instance, the maximum flow speed is needed to design the engine of a motorised capsule while the orientation of the flow is useful to predict the path of a non-motorised capsule as a function of time and landing site and to select the latter such that the capsule does not beach within a short amount of time. The variation of the sea/lake surface elevation is a key parameter in order to ensure the capsule is able to be refloated from the beach, if it beaches on purpose at a given location. The influence of the poorly constrained parameters has to be assessed by means of sensitivity analyses. The results can then be used to design the selected operation mission taking into account the worst case scenario predicted by the sensitivity analyses. The data collected by this exploration mission could be used to validate more accurately the models and improve them as well as our understanding of the physical mechanisms at play. The results of the improved models could then be used to help analysing the data collected or to design more complex exploration missions.

While there are proposals for exploration missions focusing on the surface lakes and seas, such missions are not expected to take place before years. Indeed, the surface liquid bodies are located in the high latitudes, mostly in the northern hemisphere. The last spring equinox in this hemisphere occurred on August 11, 2009. A year on Titan lasting about 29.5 years on Earth, the autumn equinox in the northern hemisphere will occur before the next exploration mission could reach Titan (which is expected to be in 2034). Direct to Earth communication with a lander being impossible at high latitudes during the winter (without an orbiter acting as a relay antenna), it was decided that the next exploration mission, DragonFly, would land in the equatorial region, as it does not include an orbiter. The main drawback of a single lander is that it explores a single location, which can be circumvented if the lander is mobile. Among the concepts presented in the literature, \cite{lorenz2000post} proposed a helicopter but, a helicopter being mechanically complex, this idea was not very successful. Fortunately, technology developments in the last decades led to the advent of drones, which are much better suited for spatial exploration. The Dragonfly mission, which consists in sending a rotorcraft on Titan was recently selected by the NASA as part of its "New Frontiers" program. While it was first imagined granting the rotorcraft with amphibian capabilities, this idea was not pursued and it was decided to focus on operations on dry land \citep{lorenz2018dragonfly}. Dragonfly is expected to land at a location and season similar to the Huygens descent in 2005 to take advantage of the data collected by the probe. Once the drone is released from the entry system and parachute, it will take advantage of its mobility to scan the area and select the more favourable landing site. After Dragonfly safely landed, it will carry on small displacements and then flights of increasing range, duration, and height. The mission aims at improving our knowledge of the organic and methanogenic cycle on Titan and investigating the subsurface ocean and/or liquid reservoirs, particularly their evolution and interactions with the surface \citep{lorenz2018dragonfly}. To this end, Dragon fly will \citep{turtle2017dragonfly}
\begin{itemize}
\item Identify the chemical components and processes producing biologically relevant compounds by sampling surface material and conducting analyses with a mass spectrometer;
\item Analyse bulk elemental surface composition by means of a neutron-activated gamma-ray spectrometer;
\item Provide measurement of the atmospheric and surface conditions as well as their diurnal and spatial variations with meteorology sensors; 
\item Characterise geologic features by imaging;
\item Perform seismic studies to detect subsurface activity and structure;
\item Measure atmospheric vertical profile near the surface;
\item Provide aerial images of surface geology.
\end{itemize}

Several Jovian and Saturnian moons house a global subsurface ocean beneath their icy surface. Such subsurface oceans are interesting candidates as potential life-supporting areas. Unfortunately, the subsurface oceans are out of reach for common exploration missions although some aspects such as their density and depth can be inferred/modelled from the data collected. Different models are required to study the habitability of such oceans. Three main approaches exist: the "follow the energy", the "follow the water", and the "follow the nutriment" approaches \citep[see, e.g.,][]{lingam2018extraterrestrial}. Those three approaches require an \textit{a priori} knowledge of the liquid motion within the ocean, which can be provided by a hydrodynamic model.

While Dragonfly could provide data about Titan's subsurface ocean by performing seismic studies, the next advance regarding the subsurface oceans of icy moons will likely come from the JUpiter ICy moons Explorer (JUICE) mission which will be launched in 2022 and should start observing the Jovian moons in 2030. This mission will first orbit Jupiter, investigating Callisto and Europa and then orbit Ganymede. While the mission focuses on Jovian moons housing a subsurface ocean, it will also study other aspects of these natural satellites. Regarding Ganymede and, to a lesser extent, Callisto, JUICE will
\begin{itemize}
\item characterise the ocean layer and detect putative subsurface water reservoirs;
\item study the physical properties of the icy crust;
\item characterise the internal mass distribution, dynamics and evolution of the interiors.
\end{itemize}
JUICE will provide the first subsurface sounding of Europa and the minimal thickness of its icy crust over the most recently active regions.

The model used to study the liquid motion in Titan's global subsurface ocean could easily be modified to study the ocean of Callisto, Europa, and Ganymede. The data collected by JUICE could be used to better constrain the model's parameters regarding the Jovian moon and validate and/or improve our model. Our model is of little interest for the preparation of the mission as an orbiter does not need an \textit{a prioiri} knowledge of the domain studied. Nevertheless, the model could be used in a retro-engineering approach: once the deformations of the surface are well constrained and the number of poorly constrained parameters is reduced, the model could be used to infer the missing parameters from the results. Parameters such as the ocean depth could not be inferred as their influence on the elevation is insignificant but others could more easily be constrained. For instance, the surface heat flux generates a constant elevation which could be isolated from the temporally varying surface deformation measured. Conducting a great number of simulations for various boundary heat fluxes could provide a first approximation of the real boundary heat fluxes.

\section*{Numerical modelling}% of Titan's liquid bodies}
Numerical modelling allows virtually experiencing situations that cannot be observed and to study the impact of various parameters. It is therefore a powerful tool for scientists studying extra-terrestrial sciences for which data are scarce and observations occasional. Each model has its advantages and drawbacks. For instance, a full 3D hydrodynamic model coupled with a meso-scale atmospheric model is able to predict the surface flow which impacts the path of a capsule but it is very costly to use to conduct a sensitivity analysis with respect to the poorly constrained parameters. On the other hand, a 2D hydrodynamic model allows studying a wide range of parameters but only predicts the depth-averaged flow. The ideal tool to study a domain where the lack of data is significant is therefore a combination of these models. The less costly model should be used to study the impact of a wide range of parameters and build a database. Once the range of variation of each parameter is reduced by comparing the results with the observations, a more complex model should be used to predict more accurately the liquid motion. Depending on the domain and data, such models may not exist or the data may not be available to constrain the parameters based on the database. 

\section*{My thesis in this context}
The objective of this thesis was as follows:\vspace{0.2cm}
\begin{adjustwidth}{25pt}{25pt}
\textit{Using the scarce available data and a numerical model to improve our knowledge of the motion of surface and subsurface liquid bodies of Titan, thereby helping design future exploration missions of this moon.}
\end{adjustwidth}
\vspace{0.1cm}
To this end, I adapted the Second generation Louvain-la-Neuve Ice ocean Model, SLIM, to extra-terrestrial conditions. The equations solved by the model were modified to take into account the impact of Titan's surface deformation on the liquid motion in the surface lakes and seas and in the subsurface ocean. Due to the lack of data, sensitivity analyses were conducted with respect to the poorly constrained parameters by means of the 2D version of SLIM. The 3D version was also modified to model the liquid motion on Titan. 3D simulations being significantly more costly, the model is used sparingly to study the interactions between the thermally driven flow and the tidal flow in the subsurface ocean (see Chapter~\ref{chap:ocean}) or is applied to a specific sub-domain such as Trevize Fretum, the strait linking Kraken and Ligeia Mare \citep[see][]{vincent2017study}. 

The liquid motion predicted in the subsurface ocean could be used by astrobiologists studying the habitability of such an ocean or to constrain the internal heat fluxes. Such applications are beyond the scope of this thesis and are therefore not further developed. Regarding the surface lakes and seas, one goal of the thesis was to provide the elevation variations and the tidal flow, which are key data for the design and planning of an exploration mission focusing on the surface liquid bodies. The lakes/seas surface elevation variations allow highlighting area(s) where the capsule could beach. The tidal flow, on the other hand, can be used to predict the path of a drifted capsule, to provide design constraints for the engine of a motorised capsule/submarine, and to schedule some aspects of the mission such as the crossing of the straits linking the basins of the northern seas. Indeed, it would be convenient to go through these straits when the tidal current is in the right direction to save energy. The Lagrangian particle tracker module (LPT) implemented in SLIM was used to predict the path of a drifted capsule. It could be modified to study the path of a motorised capsule by adding the acceleration resulting from the thrust in the cinematic equations. This could be an asset to optimize the engine operation: using the latter by phase would save energy and reduce the fatigue stress. A modified version of the LPT could easily model the influence of the engine phases on the path, hence allowing conducting a sensitivity study in order to optimize the phases distribution such as the capsule follows a defined path.

Several aspects of SLIM were modified/improved as part of this thesis. The modified model used to study the surface lakes and seas could be used on Earth if the user provides the suitable parameters. There is no other application due to the absence of significant surface liquid bodies in the solar system. The model used to study the liquid motion in the subsurface ocean, on the other hand, could be easily modified to study the ocean of other icy moons housing a subsurface ocean (although particular attention should be paid to the aspect ratio for small moons such as Enceladus). Studying those oceans could be interesting in the context of the forthcoming JUICE mission. Numerical modifications of SLIM could also be useful for terrestrial applications. For instance, its ability to solve the equations on the sphere could be used to study domains where geographic projections are not well suited\footnote{In terrestrial domains, SLIM solves the equations on a flat domain which is obtained by geographic projection of the globe's surface.}. The post-processing tools developed have already been used by other members of the SLIM Team.

Various studies focusing on the surface lakes and seas were conducted in parallel with this thesis. My work differs by the high number of parameters taken into account as well as a better spatial representation of the nearshore areas by taking advantage of the geometrical flexibility of the unstructured grids. The internal structure of Titan and its ocean were also the main topic of studies conducted in parallel with this. My work focuses on the hydrodynamics of the ocean while most studies consider the ocean as a layer while studying the internal structure of Titan or pay interest to the dissipation in the ocean but not to the flow itself.



\subsection*{Ontario Lacus}

Ontario Lacus was the first lake studied as part of this thesis (see Chapters~\ref{chap:ontario} and~\ref{chap:eigenmode}). This lake was selected as the first application for two reasons: two bathymetry maps were available and unexplained shoreline variations were observed. The tides were ruled out as an explanation shortly after the beginning of this thesis, which was confirmed by our results. The elevation field is similar to that predicted by \cite{tokano2010simulation} while the velocity field significantly differs. This is due to the use of a complex bathymetry and the better spatial resolution of our simulations, which enhanced the representation of the coastline, resulting in a better capture of its effect on the flow. The normal modes of the lake were also studied, as a seiche could explain local but significant shoreline variations, and the lake response to arbitrary wind conditions was discussed. 

The tides in Ontario Lacus cannot be observed by means of an orbiter. Indeed, the tidal range is much smaller than the vertical resolution of the state-of-the-art instruments. The shoreline variations induced by the tidal motion are also too small to be observed by a radar while Titan's atmosphere makes observation in the visible spectra harder. A spacecraft landing in the lake could easily provide temporal series of the surface elevation and velocity, an accurate bathymetry map as well as the liquid composition and atmospheric conditions. The temporal series of the surface elevation could be used to validate our numerical results (the diurnal signal should be isolated as other forcings could modify the elevation). The bathymetry map would decrease the uncertainties surrounding our results and the liquid composition and atmospheric conditions could be used to take into account additional forcings while studying the liquid motion in the lake. Unlike the tidal motion, the liquid motion associated with large scale normal modes could be observed by an orbiter, if they were resonantly forced. Indeed, the latter can result in high amplitude elevation variation, which could results in significant shoreline variations. The latter could be captured by an orbiter just as those observed by Cassini.


The numerical results can be used to constrain various aspects of a mission consisting in a capsule landing in the lake. For instance, in the light of the small tidal range, less than 0.06 m, beaching should be avoided as the tides could not refloat a beached capsule. The path of a drifted capsule as a function of time and sea landing site was studied by means of the LPT. As expected, the displacements induced by the tidal motion are small, less than 1 km over 3 Titan days (i.e. 1.5 months on Earth) except in the nearshore high speed areas where it can reach a few kilometers. A floating capsule should therefore be motorised to prevent beaching and to allow more significant displacements. In Ontario Lacus, the engine of the capsule should be designed while taking into account the wind forcing as the influence of the latter on the path could be more significant than that of the tidal motion.

The model in itself does not need to be improved to enhance our representation of the tidal motion in Ontario Lacus. Additional data allowing validating the results would be helpful but they are not expected before years. The major improvement that could be effective within a reasonable amount of time would be to couple SLIM3D with a meso-scale atmospheric model to study the air-sea interactions. Indeed, although the wind speed is weak regarding terrestrial standards (at most 10 m/s under stormy conditions), the larger atmosphere density and the smaller liquid density increase the influence of the wind, resulting in a surface stress of the same order of magnitude as on Earth. Two situations are of interest: the liquid motion under stormy conditions and under standard wind conditions. The former could be related to our study of the normal modes. If a mode with a recognizable pattern and a large amplitude develops, it could be observed by the next exploration mission. The liquid motion under standard wind conditions is useful as part of the creation and design of a lander. Due to the slow tidal motion, the wind could play a significant role in the drift of a capsule both by modifying the flow and by pushing the lander in a specific direction (this observation does not stand if the lander is a submarine). The driving effect of the wind on a capsule could be modelled by adding the acceleration resulting from this force in the cinematic equations of the LPT. Using the velocity field predicted by the 3D model would be more accurate as the surface velocity can locally significantly differ from the depth averaged velocity. The particle tracker module being 2D, the surface velocity would be used by the latter instead of the depth averaged velocity.



\subsection*{Northern seas}
The northern seas of Titan are another liquid area of high interest. The tidal motion and the liquid exchange between the two largest seas were studied by means of SLIM2D (see Chapter\ref{chap:northern}). The comparison of our results with those of \cite{tokano2014numerical} is deemed satisfactory both in term of magnitude and spatial patterns. Solar glints at the northern mouth of the strait linking Kraken and Ligeia Maria were observed by \cite{sotin2012observations}. Turbulent eddies or high velocity flows could be responsible for these glints. Therefore, SLIM3D was briefly used to study this area using the sea surface elevation and depth averaged flow as boundary conditions (see \cite{vincent2017study}). The surface speed can locally reach 0.12 m/s and small gyres were observed but their impact on the sea surface roughness was not discussed. Numerical simulations of the liquid motion due to the density gradient \citep{tokano2015sun} and wind \citep{tokano2015wind} based on the predictions of a global circulation model can also be found in the literature.

An orbiter would not be well suited to observe the tidal motion as the height variations are small (especially with respect to the deformation of Titan's surface), although the horizontal variations between low and high tides could induce shoreline variations significant enough to be observed by a radar, depending on the morphology of the surroundings. A lander would be better suited for the same reasons as in Ontario Lacus. Nevertheless, it would be more interesting to send a lander in the northern seas as the later could constrain spatial variations of the atmospheric fields (surface temperature, air moisture, wind velocity \dots) on a much larger scale than in Ontario Lacus and could provide a bathymetry for Kraken Mare and data about possible variations of the liquid composition between the basins.


The northern seas are a designated target for future exploration mission(s). Several proposals were published. At least two of them consist in sending a lander (a drifted capsule or a submarine) in the northern seas. An \textit{a priori} knowledge of the tidal motion in theses seas is an asset for the design and planning of such missions. Although the tidal range is larger than in Ontario Lacus, it is still small and unable to refloat a beached capsule. The path of a drifted capsule as a function of time and landing site was studied by means of the LPT. The capsules landing offshore drift by less than 2 km over 3 Titan days (TD) while those landing in the straits can drift up to 80 km over the same period. Their paths, however, mainly consist of back and forth motion: capsules landing during a specific period are the only one which can reach the straits outlets without being driven back. Therefore a motorised capsule would be better suited as it could travel longer distance and avoid being trapped in the straits. Other forcings such as the wind could impact the path of the capsule. The effect of the wind pushing on the superstructure of the capsule could be taken into account by adding the acceleration resulting from this force in the cinematic equations of the LPT but the influence of the wind, density spatial variations, surface heat fluxes \dots on the liquid motion should also be taken into account in the hydrodynamic model. Conducting a study of the liquid motion induced by the tides, atmospheric forcings, and density gradient is tricky as the timescale of these phenomena are significantly different while they are equally significant in terms of flow speed (at least in the middle of the seas according to \cite{tokano2015wind,tokano2015sun}). The best approach seems to consist in focusing on a specific area using an atmospheric mesoscale model to predict the wind and the surface boundary conditions for the temperature equation. The lateral boundary conditions could be given by SLIM2D and the results of \cite{tokano2015wind,tokano2015sun}. The depth averaged velocity field has already been used as boundary conditions for 3D modelling of sub-domains such as the straits linking the northern seas \citep[see][]{vincent2017study}. An advantage of this approach would be that the LPT could use the surface velocity field instead of the depth-averaged velocity field to predict the displacements of a capsule. While these velocity fields are not expected to differ significantly regarding the tidal motion, this is no longer the case when baroclinic phenomena and wind forcing are taken into account. The straits linking the basins are also a key point for the design of a motorised capsule/submarine as the speed there is one order of magnitude larger than elsewhere in the northern seas. It will therefore constrain the engine of the capsule/submarine. An \textit{a priori} knowledge of the velocity variation with respect to the depth in the straits would be an asset to design a submarine. Indeed, the velocity orientation and magnitude varies with the depth. Selecting the appropriate period and/or depth for a crossing would result in a significant energy saving and would decrease the maximum engine power needed.

It could also be interesting to take advantage of the developments of SLIM conducted in parallel with this thesis. Indeed, a member of the SLIM team, Thomas Dobbelaere, recently coupled SLIM with a wave model, SWAN (Simulating WAves Nearshore, \url{swanmodel.sourceforge.net/}, see \cite{booij1997swan}). Studying the waves of the northern seas could be interesting both as part of the design of a lander and to explain phenomena such as the solar glints and the "Magic Islands". Indeed, the capsule should be stable enough to not capsize due to the regular swell and its structure should be sized to be able to resist the small shocks with the waves. Therefore, the waves height and energy should be quantified as they are useful quantities for the design of a capsule. Regarding the "Magic Islands", four explanations were initially proposed. The tides being ruled out, the remaining suspects are sediment in suspension, rising bubbles, and small waves. Due to the lack of information, it is not possible to study the sediment transport in the area of the "Magic Islands" and rising bubbles request a different model to be studied. The third explanation could be discussed by means of a coupled SLIM-SWAN model. Such a model requires an \textit{a priori} knowledge of the wind distribution during the period studied. The wind predicted by a mesoscale atmospheric model (or a GCM, depending on the domain studied) could be used as input without coupling the atmospheric model with the SLIM-SWAN model.


Possible connection(s) with a third basin south of Kraken Mare and with Punga mare could be taken into account in order to improve our model. Although no connection is visible on the Cassini images, Punga Mare is likely connected to the northern seas as its surface is on the same equipotential as those of Kraken and Ligeia Maria. Regarding the basin south of Kraken Mare, a connection appears on ISS images but not on the radar images. This basin was therefore disregarded as our study is based on radar images. These basins could however be linked to Ligeia and/or Kraken Maria by canals too small to be observed by Cassini, by a subsurface flow, or by swamps. Taking into account the connection with these basins could modify the tidal motion as the liquid exchanges between basins have a significant impact on the tidal phase (see Chapter~\ref{chap:northern}). Nevertheless, conducting such a study without basic knowledge of the domain is pointless. Crucial parameters such as the horizontal dimensions of the canals/straits or an approximation of the volumetric flow rate are required to construct a realistic approximation of the domain.

\subsection*{Global ocean}
The penultimate chapter of this thesis focuses on the liquid motion in the subsurface ocean of Titan. Studying the liquid motion in this ocean could seem to be of little interest as models predicting the deformation of Titan or the tidal dissipation are available. Nevertheless, studying the liquid motion can help ruling out or validating assumptions about the heat fluxes at the top and bottom of the ocean and the results could be useful for astrobiologists paying attention to the ocean as a life-supporting area.

The ice shell has a significant impact on the surface elevation and the velocity field. Its effect should therefore be taken into account by scientists paying interest to the ocean hydrodynamics. Nevertheless, while a fully coupled model could be used to improve our knowledge of the fluid-structure interactions, the further insight gained would be very limited due to the poorly constrained parameters. Such a model could detect a resonance phenomenon induced by the interactions but such phenomenon seems to appear only in unrealistically shallow ocean \citep[see, e.g.,][]{hay2017numerically,tyler2014comparative}. Building such a model would be a giant with feet of clay as the accuracy gained would be rendered useless by the uncertainties in the parameters. The preliminary study also shows that the tidal and thermally driven flows strongly interacts. Taking into account disregarded phenomena such as the ice melting/water freezing (and the resulting bathymetry variations), a spatially varying bottom heat flux, and the heat produced by dissipation at the boundary and within the ocean would improve our understanding of the liquid motion within a subsurface ocean more significantly than coupling the model to study the interactions with the ice shell.

\section*{Conclusion}

Using a modified version of SLIM, an Earth-based model for geophysical and environmental flows, I studied the tides of Titan's liquid bodies. This version of SLIM is able to solve the depth-averaged shallow water equations as well as the 3D hydrostatic equations under the Boussinesq approximations on flat as well as spherical extra-terrestrial domains. The global ocean of Titan lying beneath an ice shell, the hydrodynamics equations were modified to take into account the impact of the shell deformation on the liquid motion.

The tidal motion in the surface liquid bodies of Titan was studied in order to provide useful data for the forthcoming dedicated exploration mission(s) and in the hope of explaining phenomena such as shoreline variations or "Magic Islands". The tidal motion in the surface lakes and seas of Titan is of little amplitude with respect to terrestrial standards. The surface elevation, although significantly varying depending on the lake/sea, is smaller than 0.4~m in the domains studied. The flow speed is $\mathcal{O}(10^{-2})$ m/s except in the straits linking the basins of the northern seas where it is one order of magnitude larger. The relatively weak tidal amplitude is due to the deformation of Titan's surface which damps the astronomical forcing acting on the surface lakes and seas. The same damping effect is observed on Earth, but at smaller scales. Indeed, the deformation of Titan's surface is much larger in terms of amplitude as the liquid layer is filled with water, instead of liquid metal on Earth, and the Young modulus of ice is one order of magnitude smaller than that of rocks. Therefore, the damping effect on Titan is much more significant: the attenuation factor is about 0.1 while it is about 0.7 on Earth. The various phenomena observed in the surface lakes and seas such as the shoreline variations of Ontario Lacus, the "Magic Islands" and the solar glints in the northern seas cannot be explained by the tidal motion. While excited normal mode(s) could be responsible for liquid motion whose amplitude is at least one order of magnitude larger than the tidal range in Ontario lacus, their natural period is such that only atmospheric forcings could resonantly force them. The numerical results were used to predict the path of drifted capsules. Based on the paths predicted for various landing sites and time, a motorised capsule seems better suited to avoid beaching or being trapped in small areas as well as for allowing significant displacements to take place. The results can also be used to constrain the engine of a capsule landing in the surface liquid bodies and to schedule the period at which a capsule should sail through  a strait in order to take advantage of the flow orientation. The model can be used to study sub-domains more accurately or to predict the influence of various parameters on the liquid motion and/or the path of a capsule.

 The simulations of the liquid motion within the global subsurface ocean show that the surface heat fluxes derived by \cite{kvorka2018does} do not result in unrealistic liquid motion. The resulting flow strongly interact with the tidal motion, modifying significantly the velocity orientation while the surface elevation is only slightly modified. Although the predicted surface elevation corresponds to the observed deformation of Titan's surface, the results cannot be validated by comparison with data. The ice shell lying at the top of the ocean significantly impacts the tidal motion: it slows down the tidal flow and decreases the amplitude of the surface elevation by a few percentage points. If viscous deformation significantly larger than the one expected took place, the elevation and speed would be increased by the ice shell as high frequency oscillations would appear due to the 90$^\circ$ out of phase term of the surface pressure representing the viscous deformation.
 
Modelling the tides of Titan liquid bodies is a complex task as the lack of data prevents from validating the results and leads to poorly constrained parameters. I took advantage of the current state-of-the-art knowledge of Titan and a proven Earth-based model to build an image of Titan liquid tides while taking into account the remaining uncertainties. Extra-terrestrial exploration is a scientific field suffering from many stop's and go's due to the elapsed time between the missions. Each mission is a giant leap forward which can contradict the predictions based on the observations of the previous mission(s). For instance, based on Voyager spacecrafts observations, a deep ocean of liquid hydrocarbons was assumed to take place on Titan surface while Cassini mission showed that the surface lakes and seas covered only $1.3\%$ of Titan's surface. Similarly, data from forthcoming exploration mission(s) could prove that Earth-based models are not suitable for modelling the liquid motion of Titan's liquid bodies or could infirm the values of the parameters used in this thesis. 

This thesis is based on the data and observations collected by Cassini mission, which provided a clear outline of the surface lakes and seas shoreline, and bathymetry maps for some of them. Further insight were also gained in other aspects such as the composition of the liquid filling the lakes and seas, the internal structure of Titan, and the atmospheric conditions. While some observations such as the lakes/seas shoreline are hardly questionable, others are. For instance, the bathymetry map of Ontario Lacus significantly varies depending on the method used to derive it. To address this problem, sensitivity analysis with respect to the poorly constrained parameters were conducted. I am reasonably confident about the results presented in this thesis as they compare well with those available in the literature. While the quantified values do vary with the parameters, several aspects such as the global patterns of the elevation and velocity fields are rather insensitive to the parameters; except in the northern seas where the liquid exchanges between the basins have a significant influence on the phase of the main tidal component, hence modifying the spatial and temporal variations of the elevation field.
 
